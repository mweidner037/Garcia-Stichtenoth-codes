\documentclass[12pt]{article}

\usepackage{amsmath}
\usepackage{amssymb}
\usepackage{amsthm}

\usepackage{enumerate}

  % \url{}; clickable references in pdf
\usepackage{hyperref}

  %commutative diagrams
\usepackage{pictexwd,dcpic}

  %margins
\usepackage[margin=2.1cm]{geometry}

  %custom symbols
\newcommand{\mc}[1]{\ensuremath{\mathcal{#1}}}
\newcommand{\mf}[1]{\ensuremath{\mathfrak{#1}}}
\newcommand{\mb}[1]{\ensuremath{\mathbb{#1}}}
\newcommand{\ra}{\rightarrow}
\newcommand{\xra}{\xrightarrow}
\newcommand{\epi}{\twoheadrightarrow}
\newcommand{\mono}{\hookrightarrow}
\newcommand{\normal}{\vartriangleleft}
\newcommand{\Ima}{\ensuremath{{\rm Im}}}
\newcommand{\N}{\mb{N}}
\newcommand{\Z}{\mb{Z}}
\newcommand{\Q}{\mb{Q}}
\newcommand{\R}{\mb{R}}
\newcommand{\C}{\mb{C}}
\newcommand{\F}{\mb{F}}
\newcommand{\Or}{\mc{O}}
\newcommand{\sgn}{\ensuremath{{\rm sgn}}}
\newcommand{\End}{\ensuremath{{\rm End}}}
\newcommand{\Gal}{\ensuremath{{\rm Gal}}}
\newcommand{\Frob}{\ensuremath{{\rm Frob}}}
\newcommand{\Tr}{\ensuremath{{\rm Tr}}}
\newcommand{\Spec}{\ensuremath{{\rm Spec}}}
\newcommand{\charac}{\ensuremath{{\rm char}}}
\newcommand{\legendre}[2]{\left(\frac{#1}{#2}\right)}
	
	%theorems, lemmas, etc.
\theoremstyle{plain}
\newtheorem{mythm}{Theorem}[section]
\newtheorem{mylemma}[mythm]{Lemma}
\newtheorem*{mylemma_nonum}{Lemma}
\newtheorem{myprop}[mythm]{Proposition}
\newtheorem{mycor}[mythm]{Corollary}
\theoremstyle{definition}
\newtheorem{mydef}[mythm]{Definition}
\newtheorem{myex}[mythm]{Example}
\newtheorem{myrmk}[mythm]{Remark}

%header and footer
\usepackage{fancyhdr}
\pagestyle{fancy}
\setlength{\headheight}{14.5pt}
\lhead{Matthew Weidner}
\chead{}
\rhead{Functions with Small Poles}
\lfoot{}
\cfoot{\thepage}
\rfoot{}


\begin{document}
\title{Fast Construction of Functions of Many Weights with Small Pole Degrees in the Garcia-Stichtenoth Tower}
\author{Matthew Weidner}
\date{\today}
\maketitle
\vspace{-10pt}


\section{Functions with Small Poles}
TODO: notation as in SAK+

Given $n \ge 2$ and $r \in [1, q^n - 1]$, $q \nmid r$, we wish to construct a function $f^{(n)}_r$ of weight $r$ in $F_n$, regular except for poles at $S^{(n)}_t$ ($1 \le t \le n-2$) of degree at most $q^{t-1} - 1$ and a pole at $S^{(n)}_0$ of degree at most $q^n - 1$.  It will follow that $\pi_0 x_0 f^{(n)}_r$ has weight $q^{n+1} + r$ and is regular except for poles in $S^{(n)}_{2 \rceil, n-2}$, and the sum of the maximum possible weighted pole orders at these places is at most $(n-1)q^n$.

When manipulating elements, stating equalities, or referring to the structure of elements, we will always do so in the polynomial ring $\F_{q^2}[x_0, x_0^{-1}, \dots, x_n, x_n^{-1}]$.  In particular, this allows us to refer to unique decompositions into sums of monomials or into polynomials in terms of a certain variable.  Of course, all claims concerning places will only make sense in $F_n$.

TODO: base case

Write $r = e_1 q^{n-1} + e_2 q^{n-2} + \dots + e_n$.  Let $s = e_2 q^{n-2} + e_3 q^{n-3} + \dots + e_n$, and let $f^{(n-1)}_s$ be constructed recursively.  We inductively assume
\begin{equation}\label{f_induct}
f^{(n-1)}_s = \sum_{i=0}^{q-1} a_i \frac{x_1^{r_i}}{x_0^i},
\end{equation}
where each $a_j \in F_{n-1}$ is the result of shifting some $f^{(n-2)}_{t_j}$ up by 1, and each $r_j \in [1, q-1]$ except possibly $r_0$, which satisfies $r_0 = e_2$.

Let $g$ be the result of shifting all indices in $f^{(n-1)}_s$ up by 1.  By our theory of shifting (TODO), $g$ is regular except for poles in $S^{(n)}_{2, n-2}$ of the form claimed for $f^{(n)}_r$ and poles at $S^{(n)}_{0, 1}$ of degree at most $q^{n-1} - 1$.

Using equation (\ref{f_induct}) shifted up by 1, write
\[
g = \sum_{i=0}^{q-1} b_i \frac{x_2^{r_i}}{x_1^i}.
\]
Define
\[
h = \sum_{i=0}^{q-1} b_i \frac{u_2^{r_i}}{x_1^{-e_1 + i}}.
\]
Write $h$ as a polynomial in $x_0^{-1}$ as
\[
h = \sum_{j=0}^{q-1} c_j \frac{x_1^{s_j}}{x_0^j}
\]
where $s_j$ is the highest power of $x_1$ appearing in a monomial in the coefficient of $x_0^{j}$.  Since $b_i$ contains no $x_0$'s or $x_1$'s (because it comes from an element of $F_{n-2}$ shifted up by 1), we easily have $s_0 = e_1$.

Now define $t_j$ by setting $t_0 = s_0 = 0$ and setting $t_j$, $j \in [1, q-1]$, to be the unique integer such that $t_j \equiv s_j \pmod{q-1}$, $t_j \in [1, q-1]$.  Define
\[
f^{(n)}_r = \sum_{j=0}^{q-1} c_j \frac{x_1^{t_j}}{x_0^j}.
\]

TODO

TODO: mention regularity at easy places


TODO: even or odd primes; $> 2$  




\section{Fast Construction}



\end{document}
