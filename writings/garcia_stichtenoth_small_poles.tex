\documentclass[12pt]{article}

\usepackage{amsmath}
\usepackage{amssymb}
\usepackage{amsthm}

\usepackage{enumerate}

  % \url{}; clickable references in pdf
\usepackage{hyperref}

  %commutative diagrams
\usepackage{pictexwd,dcpic}

  %margins
\usepackage[margin=2.1cm]{geometry}

  %custom symbols
\newcommand{\mc}[1]{\ensuremath{\mathcal{#1}}}
\newcommand{\mf}[1]{\ensuremath{\mathfrak{#1}}}
\newcommand{\mb}[1]{\ensuremath{\mathbb{#1}}}
\newcommand{\ra}{\rightarrow}
\newcommand{\xra}{\xrightarrow}
\newcommand{\epi}{\twoheadrightarrow}
\newcommand{\mono}{\hookrightarrow}
\newcommand{\normal}{\vartriangleleft}
\newcommand{\Ima}{\ensuremath{{\rm Im}}}
\newcommand{\N}{\mb{N}}
\newcommand{\Z}{\mb{Z}}
\newcommand{\Q}{\mb{Q}}
\newcommand{\R}{\mb{R}}
\newcommand{\C}{\mb{C}}
\newcommand{\F}{\mb{F}}
\newcommand{\Or}{\mc{O}}
\newcommand{\sgn}{\ensuremath{{\rm sgn}}}
\newcommand{\End}{\ensuremath{{\rm End}}}
\newcommand{\Gal}{\ensuremath{{\rm Gal}}}
\newcommand{\Frob}{\ensuremath{{\rm Frob}}}
\newcommand{\Tr}{\ensuremath{{\rm Tr}}}
\newcommand{\Spec}{\ensuremath{{\rm Spec}}}
\newcommand{\charac}{\ensuremath{{\rm char}}}
\newcommand{\legendre}[2]{\left(\frac{#1}{#2}\right)}
	
	%theorems, lemmas, etc.
\theoremstyle{plain}
\newtheorem{mythm}{Theorem}[section]
\newtheorem{mylemma}[mythm]{Lemma}
\newtheorem*{mylemma_nonum}{Lemma}
\newtheorem{myprop}[mythm]{Proposition}
\newtheorem{mycor}[mythm]{Corollary}
\theoremstyle{definition}
\newtheorem{mydef}[mythm]{Definition}
\newtheorem{myex}[mythm]{Example}
\newtheorem{myrmk}[mythm]{Remark}

%header and footer
\usepackage{fancyhdr}
\pagestyle{fancy}
\setlength{\headheight}{14.5pt}
\lhead{Matthew Weidner}
\chead{}
\rhead{Functions with Small Poles}
\lfoot{}
\cfoot{\thepage}
\rfoot{}


\begin{document}
\title{Fast Construction of Functions of Many Weights with Small Pole Degrees in the Garcia-Stichtenoth Tower}
\author{Matthew Weidner}
\date{\today}
\maketitle
\vspace{-10pt}


\section{Functions with Small Poles}
TODO: notation as in SAK+

Given $n \ge 2$ and $r \in [1, q^n - 1]$, $q \nmid r$, we wish to construct a function $f^{(n)}_r$ of weight $r$ in $F_n$, regular except for poles at $S^{(n)}_t$ ($1 \le t \le n-2$) of degree at most $q^{t-1} - 1$ and a pole at $S^{(n)}_0$ of degree at most $q^n - 1$.  It will follow that $\pi_0 x_0 f^{(n)}_r$ has weight $q^{n+1} + r$ and is regular except for poles in $S^{(n)}_{2 \rceil, n-2}$, and the sum of the maximum possible weighted pole orders at these places is at most $(n-1)q^n$.

When manipulating elements, stating equalities, or referring to the structure of elements, we will always do so in the polynomial ring $\F_{q^2}[x_0, x_0^{-1}, \dots, x_n, x_n^{-1}]$.  In particular, this allows us to refer to unique decompositions into sums of monomials or into polynomials in terms of a certain variable.  Of course, all claims concerning places will only make sense in $F_n$.

TODO: shifting

\subsection{Inductive Assumptions}
For a multiset $S \subset (q - 1) \times [1, n]$, we will let $\elem(S)$ denote the almost regular element we construct of weight equal to that of $\prod_{i \in S} i$.

We set $\elem(\emptyset) = 1$.

Before defining $\elem(S)$, we make several inductive assumptions, which we will take to hold for all lexicographically smaller sets $S$.  Here we let $n = \max\{S\}$.
\begin{enumerate}[(A1)]
  \item We can write
  \[
  \elem(S) = x_1^{S_1} \elem(S'[-1])[1] + \sum_a s_a \elem(T_a)[1] \frac{x_1^{c_a}{x_0^{d_a}}
  \]
  for some $T_a \subset (q- 1) \times [1, n-1]$, $s_a \in \F_q$, $c_a, d_a \in [1, q-1]$.  Here $S' = S \cap (q - 1) \times [2, n]$, and $S_1$ denotes the number of 1's in $S$.
  
  \item TODO
  
  \item TODO: above weight guarantees hold
\end{enumerate}

These assumptions hold vacuously in the base case $S = \emptyset$.



\subsection{Case \texorpdfstring{$S_2 = 0$}{S2 = 0}}
Let $S = S' \sqcup (S_1 \times 1)$ with $1 \notin S'$.  Let
\[
\elem(S'[-1]) = \elem(S'[-2])[1] + \sum_a s_a \elem(T_a)[1]\frac{x_1^{c_a}}{x_0^{d_a}}.
\]
Let $\rho: \Z \ra [1, q-1]$ be defined by $\rho(m) \equiv m \pmod{q-1}$.  Set
\begin{align*}
\elem(S)
&= x_1^{S_1}\elem(S[-2])[2] + \sum_a b_a \elem(T_a)[2]\left(\frac{x_2^{c_a}}{x_1^{\rho(d_a - S_1)}} + \sum_{i=1}^{c_a} \binom{c_a}{i} x_2^{c_a - i}\frac{x_1^{\rho(2i - d_a + S_1)}}{x_0^i} \\
&= x_1^{S_1}\elem(S'[-1])[1] + \sum_a b_a \sum_{i=1}^{c_a} \binom{c_a}{i} \elem(T_a)[2]x_2^{c_a - i}\frac{x_1^{\rho(2i - d_a + S_1)}}{x_0^i} \\
x_1^{S_1}\elem(S'[-1])[1] + \sum_a b_a \sum_{i=1}^{c_a} \elem\left(T_a[2] \sqcup ((c_a - i) \times 2) \sqcup (\rho(2i - d_a + S_1) \times 1)\right) \frac{1}{x_0^i} + TODO \\
&= 
\end{align*}



\section{Fast Construction}



\end{document}
